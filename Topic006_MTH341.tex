\documentclass[12pt, a4paper]{article}
\usepackage{amsmath}
\usepackage{amssymb}
\usepackage{amsthm}

\usepackage[text={7in,10in},centering]{geometry}

\usepackage[no-math]{fontspec}
\usepackage{polyglossia}

\setdefaultlanguage{thai}
\setotherlanguage{english}

\XeTeXlinebreaklocale "th-TH"
\XeTeXlinebreakskip = 0pt plus 1pt
\setmainfont[Scale=MatchLowercase]{TH Sarabun New}
\newfontfamily{\thaifont}[Scale=MatchLowercase]{TH Sarabun New}
\renewcommand{\baselinestretch}{1.3}

\setmainfont{TH Sarabun New}
\begin{document}
\raggedleft นางสาวชลธิชา พ่วงเฟื่อง  รหัสนักศึกษา 63090500006 \\[12pt]
\hrule\vspace{12pt}
\raggedright

\begin{enumerate}
    \item จงเเสดงว่า ถ้า $p$ เป็นจำนวนเฉพาะที่ $p\geq 5$  แล้ว $p^{2}+2$ เป็นจำนวนประกอบ\\ 
    % \begin{proof} สมมติให้ $A$ เป็นเซ็ต
    % \end{proof}
    
    
    \begin{proof}
        \begin{enumerate}สมมติให้ $p$ เป็นจำนวนเฉพาะที่ $p\geq 5$ \\
        \hspace{1cm.}พิจารณา $p=5$ และ $p^{2}\equiv  1 mod 3$\\       
        %         \begin{center}   
                
        %                 $p=7$ และ $p^{2}\equiv 1 mod 3$\\
        %                 $p=11$และ $p^{2}\equiv 1mod 3$\\
        %                 $p=13$ และ $p^{2}\equiv1 mod 3$\\
        %                 $p=17$ และ $p^{2}\equiv 1 mod 3$\\
        %                 $p=19$ และ $p^{2}\equiv 1 mod 3$\\
                        
        %         \end{center}
        % \hspace{1cm.}จะได้ว่า $p$ สามารถเขียนอยู่ในรูป $p=3k+1$ สำหรับบาง $k\in \mathbb{Z} $ \\
        % \hspace{1cm.}นั่นคือ
        %         \begin{eqnarray*}
        %             p^{2}+2
        %             &=& \left (3k+1 \right )^{2}+2\\
        %             &=& \left (9k^{2}+6k+1  \right )+2\\
        %             &=& 9k^{2}+6k+3\\
        %             &=& 3\left ( 3k^{2}+2k+1 \right )
        %         \end{eqnarray*}
        % \hspace{1cm.}เป็นจริง 
        \end{enumerate}
    \end{proof}

\end{enumerate}
\end{document}

