\documentclass[12pt, a4paper]{article}
\usepackage{amsmath}
\usepackage{amssymb}
\usepackage{amsthm}

\usepackage[text={7in,10in},centering]{geometry}

\usepackage[no-math]{fontspec}
\usepackage{polyglossia}

\setdefaultlanguage{thai}
\setotherlanguage{english}

\XeTeXlinebreaklocale "th-TH"
\XeTeXlinebreakskip = 0pt plus 1pt
\setmainfont[Scale=MatchLowercase]{TH Sarabun New}
\newfontfamily{\thaifont}[Scale=MatchLowercase]{TH Sarabun New}
\renewcommand{\baselinestretch}{1.3}

\setmainfont{TH Sarabun New}
\begin{document}
\raggedleft นางสาวชลธิชา พ่วงเฟื่อง  รหัสนักศึกษา 63090500006 \\[12pt]
\hrule\vspace{12pt}
\raggedright

\begin{enumerate}
    \item 14.Prove Chain Rule Theorem: Let $I,J$ be intervals, $g : I\rightarrow \mathbb{R} $ ,  $f : J\rightarrow \mathbb{R} $ such that $f\left ( J \right )\subseteq I$, and $ c\in J.$
    If $f$ is differentiable at $c$ and $g$ is differentiable at $f\left ( c \right )$, then the composite function $g\circ f$
    is differentiable at $c $ and $ {\left (g\circ f  \right )}'\left ( c \right )={g}'\left ( f\left ( c \right ) \right )\cdot {f}'\left ( c \right ).$
    \\ \indent \\
    % \begin{proof} สมมติให้ $A$ เป็นเซ็ต
    % \end{proof}
    เริ่มจากสมการต่อไปนี้\\
    \hspace{1cm}$$\frac{g\left (f\left ( x \right )  \right )-g\left ( f\left ( c \right ) \right )}{x-c}=\frac{g\left (f\left ( x \right )  \right )-g\left ( f\left ( c \right ) \right )}{f\left (x  \right )-f\left (c  \right )}\cdot \frac{f\left (x  \right )-f\left (c  \right ) }{x-c}$$

    เมื่อใส่ลิมิต $x$ เข้าใกล้ $a$ ทั้งสองข้าง อาจจะเกิดกรณีที่ $f\left ( x \right )-f\left ( c \right )=0$ ได้ ถึงแม้ว่า $ x\neq a $ เพื่อเลี่ยงปัญหาดังกล่าว
    จะสร้างฟังก์ชันใหม่ขึ้นมา ดังบทพิสูจน์ต่อไปนี้ \\   


    \begin{proof}
        \begin{enumerate} ให้ $b=f\left ( c \right )$ และกำหนดฟังก์ชัน $h: I\rightarrow \mathbb{R}$ โดยที่ \\
        \hspace{4cm} $$ h(y)=\left\{\begin{matrix}
            \frac{g(y)-g(b)}{y-b} ,y\neq b\\ 
            g'\left ( b \right ) ,y=b
            \end{matrix}\right.$$\\
        \hspace{1cm}จะได้ว่า  $h$ มีความต่อเนื่องที่ $b$ เพราะว่า $g$ มีอนุพันธ์ที่ $b=f\left ( c \right )$ \\
        สังเกตว่า $g(y)-g(b)=h(y)(y-b)$ ทุก $y\in I$\\
        แทน $y=f\left ( x \right )$ และ $b=f\left ( c \right )$ จะได้ว่า $g\left ( f\left ( x \right ) \right )-g\left ( f\left ( c \right ) \right )=h\left ( f(x) \right ) \left ( f\left ( x \right ) -f\left ( c \right )\right )$\\
        ดังนั้น    \begin{eqnarray*}
            \frac{g\left ( f\left ( x \right ) \right )-g\left ( f\left ( c \right ) \right )}{x-c}    &=&   h\left ( f(x) \right ) \cdot   \frac{ \left ( f\left ( x \right ) -f\left ( c \right )\right )}{x-c}\\
            \lim_{x\rightarrow c}\frac{g\left ( f\left ( x \right ) \right )-g\left ( f\left ( c \right ) \right )}{x-c}  &=&\lim_{x\rightarrow c}   h\left ( f(x) \right ) \cdot \lim_{x\rightarrow c}  \frac{ \left ( f\left ( x \right ) -f\left ( c \right )\right )}{x-c}
                \end{eqnarray*}

        โดยทฤษฎีบท จะได้ว่า
        \begin{eqnarray*}\hspace{1cm}
         \lim_{x\rightarrow c}h\left ( f\left ( x \right ) \right )=h\left ( f\left ( c \right ) \right )=h\left ( b \right )\\
        \end{eqnarray*}
         ฉะนั้น \begin{eqnarray*}\hspace{1cm}   \lim_{x\rightarrow c}\frac{g\left ( f\left ( x \right ) \right )-g\left ( f\left ( c \right ) \right )}{x-c}  =h\left ( f\left ( c \right ) \right )\cdot{ f}'\left ( c \right )={g}'\left ( f\left ( c \right ) \right )\cdot {f}'\left ( c \right )\\
        \end{eqnarray*}

        นั่นคือ \begin{eqnarray*}
        \hspace{1cm} {\left ( g\circ f \right )}' \left ( c \right )={g}'\left ( f\left ( c \right ) \right ){f}'\left ( c \right )
        \end{eqnarray*}
         
         
         
         % \hspace{1cm.}จะได้ว่า $p$ สามารถเขียนอยู่ในรูป $p=3k+1$ สำหรับบาง $k\in \mathbb{Z} $ \\
        % \hspace{1cm.}นั่นคือ
        %         \begin{eqnarray*}
        %             p^{2}+2
        %             &=& \left (3k+1 \right )^{2}+2\\
        %             &=& \left (9k^{2}+6k+1  \right )+2\\
        %             &=& 9k^{2}+6k+3\\
        %             &=& 3\left ( 3k^{2}+2k+1 \right )
        %         \end{eqnarray*}
        % \hspace{1cm.}เป็นจริง 
        \end{enumerate}
    \end{proof}

\end{enumerate}
\end{document}

